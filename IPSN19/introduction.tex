
\section{Introduction}
\label{sec:intro}
Battery operated mobile devices have limited life time and size which are determined by their source of power rather than environmental hazard that cause by such a source. Despite harvested energy is used to recharge the battery to eliminate the replacing process that added more cost and time for maintenance of these systems, this process goes very slowly and don?t solve the size, life time, and environmental hazard. large capacitors are used to replace the rechargeable battery to extended the life time of portable tiny systems for decade and reduced their sizes with no impact on the environment. Using centralized large capacitor suffers from unfairness energy distribution among device peripherals which make any hungry energy task or components to consume most of the harvested energy and that?s produce more system failure and less running time.  
Multiple small capacitors for microcontroller and each peripheral such in UFOP and Flicker allow devices to start up quickly, harvest energy more efficiently and simplify software decision making. These systems don't make effective use of excess harvested energy. Capybara attempts to use the excessive energy by assigning energy hungry tasks to be run during the burst time by configuring a chain of capacitors. This approach can suffer from limiting certain tasks for burst time and not be done in other time. So, we developed xx that allows us to store excess harvested energy, without sacrificing fast charge-up times and availability for the system and its peripherals. Also, XX  provide overvoltage protection for all system components without using voltage regulator for each component that increased the board size and add power consumption to the system. 

We evaluated xx  using Ehko with three applications (event detection, continues sensing and 



\subsection*{Contributions}

The contributions of this paper include:

\begin{compactenum}
	\item 
	\item 
	\item 
\end{compactenum}

\noind
\sysname 